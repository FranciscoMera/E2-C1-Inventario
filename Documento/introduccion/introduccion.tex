\section{Alcance del Documento}

El presente documento tiene como propósito especificar los requerimientos solicitados por \textbf{Imagina S.A. de C.V.} en el entregable pasado, así como la definición de nuevos requerimientos y su especificación utilizando las herramientas de UML para modelar el sistema \textbf{SGIP}. Así mismo servirá como apoyo para que el equipo de desarrollo, a cargo de \textbf{\lider}, lleve a cabo la implementación de los casos de uso descritos en los posteriores capítulos.

Es importante y vital, no sólo para este entregable, sino para todo el proyecto no olvidar el Statement of Work:

\begin{center}
	\textit{"vsdfv"}
\end{center}

\section{Organización del Documento}

En la primera parte del documento se específica de forma detallada de que trata el proceso así como los requerimientos obtenidos para este módulo utilizando la notación UML.

En el capítulo \ref{ch:req} se redactan aquellos requerimientos de usuario y del sistema(funcionales) que ayudarán a definir este módulo.

En el capítulo \ref{ch:glo} se presentan aquellos términos del negocio que ayudan a comprender hechos relevantes del negocio.

En el capítulo \ref{ch:Informacion}

En el capítulo \ref{ch:reglas} se presentan


En el capítulo \ref{ch:tobe} utilizando BPMN se propone la nueva forma de operar del negocio con base en las solicitudes de pedidos que cada una de las sucursales de \textbf{Imagina} tiene.


\section{Notación}

\notacion





